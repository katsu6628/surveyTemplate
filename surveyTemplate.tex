\documentclass[a4j,landscape,twocolumn]{jsarticle}
\usepackage[T1]{fontenc}
\usepackage{amsmath,amssymb}
\usepackage[dvipdfmx]{graphicx}
\usepackage{array}
\usepackage{float}
\usepackage{amsthm}
\usepackage{ascmac}
\usepackage{bm}
\usepackage[dvipdfmx]{color}
\usepackage{subcaption}
\usepackage{txfonts}				%\mathfrak{R}実部のR等
\usepackage[top=10truemm,bottom=10truemm,left=15truemm,right=15truemm]{geometry}
\usepackage{fancyhdr}
\parindent = 0pt
\newcommand{\parttitle}[1]{\begin{screen}\centering{\large{\bf{#1}}}\end{screen}}
\begin{document}
\newpage
\thispagestyle{fancy}
\lhead[]{
%% Article name and published Year

}
\rhead[]{
%% The date you read this article

}
\twocolumn[\bf{\huge
%% The title of the Article

}\\
%% Name of Authors

]
\vspace*{1mm}
\parttitle{論文の概要、どんな内容?}
%% Summary of Article

\parttitle{問題設定と解決した点、先行研究と比べてどこがすごい?}
%% Theme and Difference from previous works

\parttitle{技術や手法のキモはどこ?}
%% Main part of technology

\parttitle{どうやって有効だと検証した?}
%% Evidence of work

\newpage
\vspace*{1mm}
\parttitle{議論すべき点}
%% Discussions

\parttitle{次に読むべき論文は?}
\begin{enumerate}
%% What next
\item

\end{enumerate}
\parttitle{Appendix}
%% Figures
\begin{figure}[H]
\centering
\begin{subfigure}{0.48\columnwidth}
\centering
\includegraphics[width=\columnwidth]{}
\caption{}
\end{subfigure}
\begin{subfigure}{0.48\columnwidth}
\centering
\includegraphics[width=\columnwidth]{}
\caption{}
\end{subfigure}
\caption{}
\label{}
\end{figure}

\end{document}